\documentclass[aps,pre,twocolumn,showpacs,preprintnumbers,amsmath,amssymb]{revtex4-1}
%\documentclass[preprint,showpacs,preprintnumbers,amsmath,amssymb]{revtex4}

% Some other (several out of many) possibilities
%\documentclass[preprint,aps]{revtex4}
%\documentclass[preprint,aps,draft]{revtex4}
%\documentclass[prb]{revtex4}% Physical Review B

\usepackage{graphicx}% Include figure files
\usepackage{dcolumn}% Align table columns on decimal point
\usepackage{bm}% bold math 

\usepackage[latin1]{inputenc}
\usepackage{amsfonts}
\usepackage{amsmath}
\usepackage{amssymb}
\usepackage{amsthm}
\usepackage{bbold}
\usepackage[dvipsnames]{xcolor}
\usepackage[american]{babel}
\usepackage[T1]{fontenc}
\usepackage{longtable}
%\usepackage[dvips]{graphicx}
\usepackage{xspace}
\usepackage{bbm}
\usepackage[all]{xy}
%\usepackage{slashbox}
%\usepackage[justification=centering]{caption}
\providecommand{\begeq}[1]{\begin{equation}#1\end{equation}}
\DeclareMathOperator{\tr}{tr}
\providecommand{\norm}[1]{\lVert#1\rVert}
\newtheorem{theorem}{Theorem}
\newtheorem{lemma}{Lemma}
\newtheorem{defi}{Definition}
\newtheorem{rem}{Remark}
\newtheorem{conj}{Conjecture}
\newtheorem{prop}{Proposition}
\DeclareMathOperator{\con}{cond}
\DeclareMathOperator{\diag}{diag}
\newcolumntype{C}[1]{>{\centering\arraybackslash}p{#1}}

\newcommand{\michael}[1]{\textcolor{MidnightBlue}{#1}}

%Try to avoid widows and orphans
%\usepackage[all]{nowidow}
\clubpenalty = 10000
\widowpenalty = 10000
\displaywidowpenalty = 10000

\begin{document}

%\preprint{APS/123-QED}

\title{Accurate Sampling with Noisy Forces from Approximate Computing}% Force line breaks with \\

\author{Varadarajan Rengaraj}
%\email{rengaraj@campus.uni-paderborn.de}
\affiliation{Dynamics of Condensed Matter and Department of Computer Science, Paderborn University, Warburger Str. 100, D-33098 Paderborn, Germany}

\author{Michael Lass}
%\email{michael.lass@uni-paderborn.de}
\affiliation{Department of Computer Science and Paderborn Center for Parallel Computing, Paderborn University, Warburger Str. 100, D-33098 Paderborn, Germany}

\author{Christian Plessl}
%\email{christian.plessl@uni-paderborn.de}
\affiliation{Department of Computer Science and Paderborn Center for Parallel Computing, Paderborn University, Warburger Str. 100, D-33098 Paderborn, Germany}

\author{Thomas D. K\"uhne}
\email{tdkuehne@mail.upb.de}
\affiliation{Dynamics of Condensed Matter and Center for Sustainable Systems Design, Chair of Theoretical Chemistry, University of Paderborn, Warburger Str. 100, D-33098 Paderborn, Germany}

\date{\today}% It is always \today, today,
             %  but any date may be explicitly specified


\begin{abstract}
In scientific computing, the acceleration of atomistic computer simulations by means of custom hardware is finding ever growing application. 
%Accelerating atomistic computer simulations using hardware acceleration is finding ever growing application in scientific computing. 
A major limitation, however, is that the high efficiency in terms of performance and low power consumption entails the massive usage of low-precision computing units. Here, based on the approximate computing paradigm, we present an algorithmic method to rigorously compensate for numerical inaccuracies due to low-accuracy arithmetic operations, yet still obtaining exact expectation values using a properly modified Langevin-type equation. %By this, we lay the foundation for the development of efficient MD accelerators.
\end{abstract}

%A Valid PACS numbers may be entered using the \verb+\pacs{#1}+ command.
\pacs{31.15.-p, 31.15.Ew, 71.15.-m, 71.15.Pd}% PACS, the Physics and Astronomy
                             % Classification Scheme.
\keywords{}%Use showkeys class option if keyword
                              %display desired
\maketitle



\section{Introduction}
Molecular dynamics (MD) is a very powerful and widely used technique to study thermodynamic equilibrium properties, as well as the real-time dynamics of complex systems made up of interacting atoms \cite{AlderWainwright1957}. This is done by numerically solving Newton's equations of motion in a time-discretized fashion via computing the nuclear forces of all atoms at every time step \cite{RahmanMD}. Computing these forces by analytically differentiating the interatomic potential with respect to the nuclear coordinates is computationally rather expensive, which is particularly true for electronic structure based \textit{ab-initio} MD simulations \cite{CPMD, CPMD_TDK, PayneRMP, WIRES_TDK}. 

For a long time newly developed microchips became faster and more efficient over time due to new manufacturing processes and shrinking transistor sizes. However, this development slowly comes to an end as scaling down the structures of silicon based chips becomes more and more difficult. The focus therefore shifts towards making efficient use of the available technology. Hence, beside algorithmic developments \cite{MTS, Snir, GSE, Shaw, VerletCell, pSHAKE, John, Prodan}, there have been numerous custom computing efforts in this area to increase the efficiency of MD simulations by means of hardware acceleration, which we take up in this work. Examples of the latter are MD implementations on graphics processing units (GPUs) \cite{HOOMD, NAMD, OpenMM, HalMD, Lammps, Amber, Gromacs}, field-programmable gate arrays (FPGAs) \cite{HerbordtI, HerbordtII}, and application-specific integrated circuits (ASICs) \cite{AntonI, AntonII}. 
%especially the ones based on graphics processing unit (GPU) and field-programmable gate array (FPGA). 
While the use of GPUs for scientific applications is relatively widespread \cite{GPUcomp,Binder,Weigel}, the use of ASICs \cite{QCDScience, QCDOC, GrapeScience, Grape} and FPGAs is less common \cite{JanusI, JanusII, Convey, FDTD, Kenter, Galerkin}, but gained attention over the last years.
%Another approach is the use of accelerator hardware in the form of graphics processing units (GPUs), application-specific integrated circuits (ASICs) or field-programmable gate arrays (FPGAs). While the use of GPUs for scientific applications is relatively wide-spread, the use of ASICs and FPGAs is less common but gained attention over the last years.
In general, to maximize the computational power for a given silicon area, or equivalently minimize the power-consumption per arithmetic operation, more and more computing units are replaced with lower-precision units. This trend is mostly driven by market considerations of the gaming and artificial intelligence industries, which are the target consumers of hardware accelerators and naturally do not absolutely rely on full computing accuracy. 

%Microchips sizes of FPGA and GPU, are on a constant decline to accommodate more transistors but it also makes the transistors susceptible to both temporary and permanent failures. These hardware faults occasionally propagate to the software and considering this aspect, there is a renewed interest in approximate computing that can be applied in the software to give us the outputs that does not diverge too much from the ideal outputs. Approximate computing also ensures that the portion of investment needed in detecting the hardware faults, avoidance and recovery is avoided.

In our approach, we are going to present here, to effectively make use of low-precision special-purpose hardware for general-purpose scientific computing is based on the approximate computing (AC) paradigm~\cite{KlavikMalossiBekasEtAl2014, PlesslAC}. The general research goal of AC is to devise and explore ingenious techniques to relax the exactness of a calculation to facilitate the design of more powerful and/or more efficient computer systems. However, in scientific computing, where the exactness of all computed results is of paramount importance, attenuating accuracy requirements is not an option. Yet, we will demonstrate that it is nevertheless possible to rigorously compensate for numerical inaccuracies due to low-accuracy arithmetic operations and still obtain exact expectation values as obtained by ensemble averages of a properly modified Langevin equation. 

%In this paper, we demonstrate the resilience of MD, simulating the use of low-precision data types. By this, we lay the foundation for the development of efficient MD accelerators.

The remainder of the paper is organized as follows. In section II we revisit the basic principles of AC before introducing our modified Langevin equation in section III. Thereafter, in section IV, we describe the computational details of computational experiments. Our results are presented and discussed in Section V before concluding the paper in Section VI.

%In this paper, we describe one such technique that relaxes the exactness of the output and we explore to what extent it diverges from the ideal output.


\section{Approximate Computing}
\subsection{Common number formats and their support in hardware}

\michael{In recent time we see increased use of floating-point variants
different to single-precision and double-precision. Table~\ref{tab:float} lists
some commonly used data types, and the number of bits used to represent the
exponent and the mantissa (incl. the implicit bit).}

\begin{table}
  \caption{Common floating-point types}
  \label{tab:float}
  \begin{tabular}{lrr}
    Type & exponent bits & mantissa bits \\
    \hline
    Quadruple-precision & 15 & 113 \\
    Double-precision & 11 & 53 \\
    Single-precision & 8 & 24 \\
    Half-precision & 5 & 11 \\
    Bfloat16 & 8 & 8 \\
  \end{tabular}
\end{table}

\michael{Hardware support for these formats are typically available for
single-precision and double-precision. Due to the spread of machine learning
applications, some modern GPUs targeting the data center start supporting
half-precision as well, doubling the peak performance compared to
single-precision and quadrupling it compared to double-precision
arithmetics~\cite{tesla}. Due to the low number of exponent bits, half-precision
however only provides a very limited dynamic range. In contrast, bfloat16
provides the same dynamic range as single-precision, and just reduces precision.
It is currently supported by Google's Tensor Processing Units (TPU)~\cite{tpu}
and support is announced for future Intel Xeon processors~\cite{xeon}.}

\michael{FPGAs as a platform for custom-built accelerator designs can make
effective use of all of these and also entirely custom number formats. The
benefit of using low-precision types for peak performance and resource
utilization are not as foreseeable as on fixed hardware architectures but
instead depend on how efficiently the given hardware specification can be mapped
to hardware elements on the FPGA. However, similar speedups as for CPUs and GPUs
can often be achieved.}

\michael{In addition to floating-point formats, also fixed-point representations
can be used. Here, numbers are stored as an integer of fixed size with a
predefined scaling factor. Calculations are thereby performed using integer
arithmetic. On CPUs and GPUs only certain models can perform integer operations
with a peak performance similar to floating-point, depending on the
capabilities of the vector units / stream processors. FPGAs typically can
perform integer operations with performance similar to or higher than
floating-point.}

\michael{Due to the high flexibility of FPGAs with respect to different data
formats and the possible use of entirely custom data types, we see them as main
target technology for our work. For this reason, we consider both floating-point
and fixed-point arithmetic in the following.}

%\input{sections/foundations-chem}

\section{Methodology}
To demonstrate the concept of approximate computing, we introduce white noise to the interatomic forces that are computed while running the MD simulation. In this section, we describe in detail on how we introduce the noise to mimic in software the behaviour that would happen when running the MD on the actual FPGA or GPU hardware with reduced numerical precision. We classify the computational errors into two types 1. Fixed-point error 2. Floating-point error. Assuming that $\textbf{F}_{I}$ are the exact and $\textbf{F}_{I}^{N}$ the noisy forces from a MD simulations with low precision on a FPGA for instance, fixed-point errors can by modelled by 
\begin{equation}
%\begin{pmatrix}
%\textbf{F}_{I}^{N(x)}\\ 
%\textbf{F}_{I}^{N(y)}\\ 
%\textbf{F}_{I}^{N(z)}\\ 
%\end{pmatrix} = 
\textbf{F}_{I}^{N}=
\begin{pmatrix}
\text{F}_{I}^{x}\\ 
\text{F}_{I}^{y}\\ 
\text{F}_{I}^{z}\\ 
\end{pmatrix} + 
\begin{pmatrix}
c_{1} \times 10^{-\beta }\\ 
c_{2} \times 10^{-\beta }\\ 
c_{3} \times 10^{-\beta }\\ 
\end{pmatrix}
,
\end{equation}
whereas floating-point errors are described by
\begin{equation}
%\begin{pmatrix}
%\textbf{F}_{I}^{N(x)}\\ 
%\textbf{F}_{I}^{N(y)}\\ 
%\textbf{F}_{I}^{N(z)}\\ 
%\end{pmatrix} = 
\textbf{F}_{I}^{N} = 
\begin{pmatrix}
\text{F}_{I}^{x} \times 10^{-\alpha_1}\\ 
\text{F}_{I}^{y} \times 10^{-\alpha_2}\\ 
\text{F}_{I}^{z} \times 10^{-\alpha_3}\\ 
\end{pmatrix} + 
\begin{pmatrix}
c_{1} \times 10^{-(\alpha_1+\beta)}\\ 
c_{2} \times 10^{-(\alpha_2+\beta)}\\ 
c_{3} \times 10^{-(\alpha_3+\beta)}\\ 
\end{pmatrix}
.
\end{equation}
Therein, $c_1$, $c_2$ and $c_3$ are random values chosen in the range [-0.5, 0.5], whereas $\text{F}_{I}^{x}$, $\text{F}_{I}^{y}$ and $\text{F}_{I}^{x}$ are the individual force components of $\textbf{F}_{I}$, respectively. The floating-point scaling factor is denoted as $\alpha$ and the magnitude of the applied noise by \(\beta\).
%as discussed in detail in the computational details section. \(\textbf{F}_{I}\) is the exact consistent force and \(\textbf{F}_{I}^{FPGA}\) is the force computed when MD is run on the FPGA.

For to purpose to rigorously correct for the errors due to the introduced numerical noise we employ a properly modified Langevin equation. In particular, we model the force as obtained by a low-precision computation on a GPU or FPGA-based accelerator as 
%we assume at this point a computational error \(\mathbf{\Xi }_{I}^{N}\) is added to the force that is computed and the force that we get at the output is not the exact force \(\textbf{F}_{I}\) but an approximation
\begin{equation} \label{fFPGA}
\textbf{F}_{I}^{N} = \textbf{F}_{I} + \mathbf{\Xi }_{I}^{N}, 
\end{equation}
where $\mathbf{\Xi }_{I}^{N}$ is an additive white noise for which
\begin{equation} \label{CrossCorr}
 \left \langle \textbf{F}_{I}\left ( 0 \right ) \mathbf{\Xi } _{I}^{N}\left ( t \right )\right \rangle \cong  0
\end{equation}
holds. Given that $\mathbf{\Xi }_{I}^{N}$ is unbiased, which is in our case is true by its very definition, it is nevertheless possible to accurately sample the Boltzmann distribution by means of a Langevin-type equation \cite{Krajewski,Richters,Karhan}, which in its general form reads as
%Fortunately in our case, it is still possible to accurately obtain Boltzmann sampling by means of a modified Langevin equation 
\begin{equation} \label{LangevinEq}
M_{I}\ddot{\textbf{R}}_{I}=\textbf{F}_{I}+\mathbf{\Xi }_{I}^{N}-\gamma _{N}M_{I}\dot{\textbf{R}}_{I}, 
\end{equation}
where $\dot{\textbf{R}}_{I}$ are the nuclear coordinates (the dot denotes time derivative), $M_I$ are the nuclear masses and $\gamma _{N}$ is a damping coefficient, 
%where \(\textbf{R}_{I}\) are the positions of the atoms, \(M_{I}\) the corresponding atomic nuclear masses and \(\gamma _{N}\) is a friction coefficient 
which is chosen to compensate for \(\mathbf{\Xi }_{I}^{N}\). The latter, in order to guarantee an accurate canonical sampling, has to obey
%\begin{equation}
% \left \langle \textbf{F}_{I}\left ( 0 \right ) \mathbf{\Xi } _{I}^{N}\left ( t \right )\right \rangle \cong  0,
%\end{equation}
the fluctuation-dissipation theorem
\begin{equation}
\left \langle \mathbf{\Xi }_{I}^{N}\left ( 0 \right ) \mathbf{\Xi }_{I}^{N}\left ( t \right ) \right \rangle \cong  2 \gamma_{N} M_I k_{B} T  \delta \left ( t \right ), 
\end{equation}  
where $k_B$ is Boltzmann's constant, and $T$ is the desired temperature. 

Substituting Eq.~\ref{fFPGA} into Eq.~\ref{LangevinEq} results in the desired modified Langevin equation 
\begin{equation} \label{modLangevin}
M_{I}\ddot{\textbf{R}}_{I} = \textbf{F}_{I}^{N}-\gamma _{N}M_{I}\dot{\textbf{R}}_{I}, 
\end{equation}
which will be used throughout the remaining of this paper.

%We further present in this paper the method to effectively compensate for the additive noise introduced on the force using the existing langevin dynamics framework. 

 
\section{Computational details}
To demonstrate our method as described above, we have implemented it in the Frontiers in simulation technology (force field implementation) code which is part of the publicly available suite of programs CP2k \cite{cp2kwebsite}. We configured to run the langevin dynamics for the Silicon (Si) atom with a time step of 1 femtosecond (fs) at a temperature of 3000 K. The Empirical Interatomic Potential (EIP) calculation model for our MD simulation is Bazant potentials. The total number of Si atoms used for the MD simulation is thousand with a cubic cell length of 27.155 Angstrom. 

Langevin dynamics configured for the above settings have been run with frictional coefficient \(\gamma\) assigned a value equal to 1/1000 \(fs^{-1}\). This simulation run is considered as the reference with which the computational error introduced simulations are compared. As described in the Methodology section, computational errors are classified as fixed point error and floating point error. With necessary changes incorporated to the cp2k software, two different builds for the two variants of errors have been made and Langevin dynamics were performed on those builds. Three different cases of fixed point errors were tested and for the equation representing fixed point error, the corresponding \(\beta\) value that was used were 1, 2 and 3. Similarly three different cases of floating point errors were tested and for the equation representing floating point error, the corresponding \(\beta\) value that was used were 0, 1, 2.

As discussed earlier in the methodology section, a separate frictional coefficient is necessary to compensate for the computational error that was  introduced in the builds. This frictional coefficient is managed through the parameter called shadow \(\gamma\) and this parameter is available in the langevin section of the cp2k software.   

The following tables lists down shadow \(\gamma\) values used for fixed point and floating point errors.  Units for \(\gamma\) and shadow \(\gamma\) is \(fs^{-1}\). As described in this paper \cite{ShadowGammaEstimate} an approximate value of shadow \(\gamma\) can be computed by integrating the autocorrelation function of the additive white noise. \\\\\\


\textbf{Table 1.} Shown are the shadow \(\gamma\) values for fixed point errors. \\ \\
\begin{table}[h!]
\begin{tabular}{|l|l|}
\hline
\textit{\(\beta\) } & \textit{Shadow \(\gamma\)} \\ \hline
1             & 0.0004                \\ \hline
2             & 0.000009              \\ \hline
3             & 0.0000009             \\ \hline
\end{tabular}
\end{table}


\textbf{Table 2.} Shown are the shadow \(\gamma\) values for floating point errors.\\

\begin{table}[h!]
\begin{tabular}{|l|l|}
\hline
\textit{\(\beta\) } & \textit{Shadow \(\gamma\)} \\ \hline
0             & 0.00025               \\ \hline
1             & 0.000005              \\ \hline
2             & 0.000005              \\ \hline
\end{tabular}
\end{table}

 
\section{Results and Discussion}
In Fig. 1 and Fig. 2 we compare the pair correlation functions g(r) calculated with noisy forces and those evaluated with the standard approach. In Fig. 1 we show the pair correlation functions g(r) calculated with noisy forces generated from fixed point errors. In Fig. 2 we show the pair correlation functions g(r) calculated with noisy forces generated with floating point errors. We see from both Fig. 1 and Fig. 2, that the results are in agreement with our reference calculations and the use of noisy forces does not degrade the local structure and dynamics of the standard system. 

\begin{figure}[h!]%[!htpb,floatfix]
\begin{center}
\includegraphics[width=0.475\textwidth]
{figures/fixedpoint.pdf}
\end{center}
\caption{\label{Fig1}
Pair correlation function for (a) liquid silicon (3000~K) in red and (b) liquid silicon (3000~K) with noisy forces introduced by fixed point errors corresponding to the values \(\left ( c.10^{-1 }\right ) \)(blue), \(\left ( c.10^{-2 }\right ) \)(yellow) and \(\left ( c.10^{-3 } \right ) \)(green).
} \end{figure}

\begin{figure}[h!]%[!htpb,floatfix]
\begin{center}
\includegraphics[width=0.475\textwidth]
{figures/floatingpoint.pdf}
\end{center}
\caption{\label{Fig2}
Pair correlation function for (a) liquid silicon (3000~K) in red and (b) liquid silicon (3000~K) with noisy forces introduced by floating point errors corresponding to the values \(c.10^{-(\alpha)}\)(blue), \(c.10^{-(\alpha+1)}\)(yellow) and \(c.10^{-(\alpha+2)}\)(green).
} \end{figure}

\begin{figure}[h!]%[!htpb,floatfix]
\begin{center}
\includegraphics[width=0.475\textwidth]
{figures/maxwelldistribution.pdf}
\end{center}
\caption{\label{Fig3}
Statistical properties in a 1000 atoms liquid Si simulation at 3000~K. (a) The ionic kinetic energy distributions (line) is compared with the exact Maxwell distribution (circles).
} \end{figure}

In Fig. 3, the ionic kinetic energy distribution calculated with noisy forces is compared with the exact Maxwell distribution line and it can be seen that not only the average energy is correct but also its fluctuations follow the Maxwell distribution. 

\begin{figure}[h!]%[!htpb,floatfix]
\begin{center}
\includegraphics[width=0.475\textwidth]
{figures/force_autocorrelation.pdf}
\end{center}
\caption{\label{Fig4}
The Autocorrelation of the noisy force \(
\left \langle \textbf{F}_{I}^{FPGA}\left ( 0 \right ) \textbf{F}_{I}^{FPGA}\left ( t \right )\right \rangle \)(line) is compared with the autocorrelation of the exact force \( \left \langle \textbf{F}_{I}\left ( 0 \right ) \textbf{F}_{I}\left ( t \right )\right \rangle \)(circles). 
} \end{figure}

The autocorrelation of the noisy force is expanded by the following equation, 

\begin{subequations}
\begin{equation}
\begin{split}
\left \langle \textbf{F}_{I}^{FPGA}\left ( 0 \right )\textbf{F}_{I}^{FPGA}\left ( t \right )\right \rangle = \linebreak \\ \left \langle \left ( \textbf{F}_{I}\left ( 0 \right ) + \mathbf{\Xi } _{I}^{N}\left(0 \right )\right) \left( \textbf{F}_{I}\left ( t \right )+\mathbf{\Xi } _{I}^{N}\left ( t \right )\right) \right \rangle
\end{split}
\end{equation}

\begin{equation}\label{eq:1}
\begin{split}
\left \langle \textbf{F}_{I}^{FPGA}\left ( 0 \right )\textbf{F}_{I}^{FPGA}\left ( t \right )\right \rangle = \left \langle \textbf{F}_{I}\left ( 0 \right ) \textbf{F}_{I}\left ( t \right )\right \rangle + \linebreak \\\left \langle \textbf{F}_{I}\left ( 0 \right ) \mathbf{\Xi } _{I}^{N}\left(t \right )\right \rangle +  \left \langle \textbf{F}_{I}\left ( t \right ) \mathbf{\Xi } _{I}^{N}\left(0 \right )\right \rangle + \left \langle \mathbf{\Xi } _{I}^{N}\left(0 \right ) \mathbf{\Xi } _{I}^{N}\left(t \right )\right \rangle
\end{split}
\end{equation}
\end{subequations}

In Eq. (\ref{eq:1}), the cross correlation terms between exact force and the additive white noise becomes zero, thereby giving us an equation where the autocorrelation of noisy force is equal to the autocorrelation of the exact force. From Fig. 4, we prove this behavior where the autocorrelation of the noisy force is compared with the autocorrelation of the exact force. 

If the random value that was used to generate the computational error is chosen in the range [0,1] instead of [-0.5,0.5], we encounter a phenomenon called "Flying Ice Cube" \cite{flyingIceCube} during our MD simulations.  


\section{Summary}
To summarize, in this paper, we described the method to compensate for the computational errors that are introduced when running the MD code in FPGA or GPU using approximate computing technique.  



\begin{acknowledgments}
The authors would like to thank the Paderborn Center for Parallel Computing (PC2) for computing time on OCuLUS and FPGA-based supercomputer NOCTUA. Funding from the Paderborn University's research award for ``Green IT'' is kindly acknowledged. This project has received funding from the European Research Council (ERC) under the European Union's Horizon 2020 research and innovation programme (grant agreement No 716142).
\end{acknowledgments}

\bibliography{report}

\end{document}
