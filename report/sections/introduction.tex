Molecular dynamics (MD) is a standard technique to study the movement of atoms in a substance over time. It involves computing the forces on all atoms for every time step as a product of the bonded and non- bonded interactions. This is done by numerically solving the Newton's law of motions and update the parameters such as velocity and position of each atom. Computing the forces from non-bonded interactions is computationally expensive and our conventional multicore processors falls behind on the computational requirements. There has been numerous efforts going in this area to accelerate the MD simulations especially the ones based on graphics processing unit (GPU) and field-programmable gate array (FPGA).   
Microchips sizes of FPGA and GPU, are on a constant decline to accommodate more transistors but it also makes the transistors susceptible to both temporary and permanent failures. These hardware faults occasionally propagate to the software and considering this aspect, there is a renewed interest in approximate computing that can be applied in the software to give us the outputs that does not diverge too much from the ideal outputs. Approximate computing also ensures that the portion of investment needed in detecting the hardware faults, avoidance and recovery is avoided. The research goal of approximate computing is to explore techniques to gain more efficiency by relaxing the exactness of calculated outputs compared to the ideal outputs. In this paper, we describe one such technique that relaxes the exactness of the output and we explore to what extent it diverges from the ideal output.
