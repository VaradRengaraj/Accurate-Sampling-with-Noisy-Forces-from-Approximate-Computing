To demonstrate our approach we have implemented it in the CP2K suite of programs \cite{cp2k}. More precisely, we have conducted MD simulations of liquid Silicon (Si) at 3000~K using the environment-dependent interatomic potential of Bazant et al. \cite{EIP1,EIP2}. 
%The Empirical Interatomic Potential (EIP) calculation model for our MD simulation is Bazant potentials. The total number of Si atoms used for the MD simulation is thousand with a cubic cell length of 27.155 Angstrom. 
All simulations consisted of 1000 Si atoms in a 3D-periodic cubic box of length 27.155~\AA. Using the algorithm of Ricci and Ciccotti \cite{Ricci}, Eq.~\ref{LangevinEq} was integrated with a discretized timestep of 1.0~fs with $\gamma_N = 0.001~fs$^{-1}$. 
%Langevin dynamics configured for the above settings have been run with frictional coefficient \(\gamma\) assigned a value equal to 1/1000 \(fs^{-1}\). This simulation run is considered as the reference with which the computational error introduced simulations are compared. As described in the Methodology section, computational errors are classified as fixed point error and floating point error. With necessary changes incorporated to the cp2k software, two different builds for the two variants of errors have been made and Langevin dynamics were performed on those builds. 

Whereas the latter settings were used to compute our reference data, in total six different cases of fixed-point and floating-point errors were investigated by varying the exponent $\beta$ between 0 (huge noise) and 3 (tiny noise). %tested and for the equation representing fixed point error, the corresponding \(\beta\) value that was used were 1, 2 and 3. Similarly three different cases of floating point errors were tested and for the equation representing floating point error, the corresponding \(\beta\) value that was used were 0, 1, 2.
As already alluded to above, the additive white noise is compensated by means of Eq.~\ref{modLangevin} by varying $\gamma_N$ on-the-fly using a Berendsen-like algorithm till the equipartition theorem
\begin{equation}
\left\langle \frac{1}{2} M_I \dot{\textbf{R}}_{I} \right\rangle = \frac{3}{2} k_B T
\end{equation}
is satisfied \cite{Berendsen,TDKwater,TDKrev}. Alternatively, $\gamma_N$ can be computed by integrating the autocorrelation function of the additive white noise \cite{RZK}.
%As discussed earlier in the methodology section, a separate frictional coefficient is necessary to compensate for the computational error that was  introduced in the builds. This frictional coefficient is managed through the parameter called shadow \(\gamma\) and this parameter is available in the langevin section of the cp2k software.   
In Table~\ref{tab:gamma} the so determined values of \textit{\(\gamma_N^{fix}\)} for fixed-point and \textit{\(\gamma_N^{float}\)} for floating-point errors are reported as a function of \textit{\(\beta\)}. %Units for \(\gamma\) and shadow \(\gamma\) is \(fs^{-1}\). 
%\begin{table}[h!]
%\caption{Shadow \(\gamma\) values for the fixed point errors.}
%\begin{tabular}{|l|l|l|}
%\hline
%\textit{\(\beta\)} & \textit{\(\gamma_N^{fix}\)} & \textit{\(\gamma_N^{float}\)} \\ \hline
%0 & & 0.00025 \\ \hline
%1             & 0.0004  & 0.000005              \\ \hline
%2             & 0.000009    & 0.000005          \\ \hline
%3             & 0.0000009 &            \\ \hline
%\end{tabular}
%\end{table}
\begin{table}
  \caption{Values for \textit{\(\gamma_N^{fix}\)} and \textit{\(\gamma_N^{float}\)} as a function of \textit{\(\beta\)}.}
  \label{tab:gamma}
  \begin{tabular}{lrr}
    \textit{\(\beta\)} & \textit{\(\gamma_N^{fix}\)} & \textit{\(\gamma_N^{float}\)} \\
    \hline
    0 &           & 0.00025  \\
    1 & 0.0004    & 0.000005 \\
    2 & 0.000009  & 0.000005 \\
    3 & 0.0000009 & 
  \end{tabular}
\end{table}

%\begin{table}[h!]
%\caption{Shadow \(\gamma\) values for the fixed point errors.}
%\begin{tabular}{|l|l|}
%\hline
%\textit{\(\beta\) } & \textit{\(\gamma_N^{}\)} \\ \hline
%1             & 0.0004                \\ \hline
%2             & 0.000009              \\ \hline
%3             & 0.0000009             \\ \hline
%\end{tabular}
%\end{table}

%\begin{table}[h!]
%\caption{Shadow \(\gamma\) values for the floating point errors.}
%\begin{tabular}{|l|l|}
%\hline
%\textit{\(\beta\) } & \textit{\(\gamma_N\)} \\ \hline
%0             & 0.00025               \\ \hline
%1             & 0.000005              \\ \hline
%2             & 0.000005              \\ \hline
%\end{tabular}
%\end{table}
