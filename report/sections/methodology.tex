To demonstrate the concept of approximate computing, we introduce white noise to the interatomic forces that are computed while running the MD simulation. In this section, we describe in detail on how we introduce the noise to mimic in software the behaviour that would happen when running the MD on the actual FPGA or GPU hardware with reduced numerical precision. We classify the computational errors into two types 1. Fixed-point error 2. Floating-point error. Assuming that $\textbf{F}_{I}$ are the exact and $\textbf{F}_{I}^{N}$ the noisy forces from a MD simulations with low precision on a FPGA for instance, fixed-point errors can by modelled by 
\begin{equation}
%\begin{pmatrix}
%\textbf{F}_{I}^{N(x)}\\ 
%\textbf{F}_{I}^{N(y)}\\ 
%\textbf{F}_{I}^{N(z)}\\ 
%\end{pmatrix} = 
\textbf{F}_{I}^{N}=
\begin{pmatrix}
\text{F}_{I}^{x}\\ 
\text{F}_{I}^{y}\\ 
\text{F}_{I}^{z}\\ 
\end{pmatrix} + 
\begin{pmatrix}
c_{1} \times 10^{-\beta }\\ 
c_{2} \times 10^{-\beta }\\ 
c_{3} \times 10^{-\beta }\\ 
\end{pmatrix}
,
\end{equation}
whereas floating-point errors are described by
\begin{equation}
%\begin{pmatrix}
%\textbf{F}_{I}^{N(x)}\\ 
%\textbf{F}_{I}^{N(y)}\\ 
%\textbf{F}_{I}^{N(z)}\\ 
%\end{pmatrix} = 
\textbf{F}_{I}^{N} = 
\begin{pmatrix}
\text{F}_{I}^{x} \times 10^{-\alpha_1}\\ 
\text{F}_{I}^{y} \times 10^{-\alpha_2}\\ 
\text{F}_{I}^{z} \times 10^{-\alpha_3}\\ 
\end{pmatrix} + 
\begin{pmatrix}
c_{1} \times 10^{-(\alpha_1+\beta)}\\ 
c_{2} \times 10^{-(\alpha_2+\beta)}\\ 
c_{3} \times 10^{-(\alpha_3+\beta)}\\ 
\end{pmatrix}
.
\end{equation}
Therein, $c_1$, $c_2$ and $c_3$ are random values chosen in the range [-0.5, 0.5], whereas $\text{F}_{I}^{x}$, $\text{F}_{I}^{y}$ and $\text{F}_{I}^{x}$ are the individual force components of $\textbf{F}_{I}$, respectively. The floating-point scaling factor is denoted as $\alpha$ and the magnitude of the applied noise by \(\beta\).
%as discussed in detail in the computational details section. \(\textbf{F}_{I}\) is the exact consistent force and \(\textbf{F}_{I}^{FPGA}\) is the force computed when MD is run on the FPGA.

For to purpose to rigorously correct for the errors due to the introduced numerical noise we employ a properly modified Langevin equation. In particular, we model the force as obtained by a low-precision computation on a GPU or FPGA-based accelerator as 
%we assume at this point a computational error \(\mathbf{\Xi }_{I}^{N}\) is added to the force that is computed and the force that we get at the output is not the exact force \(\textbf{F}_{I}\) but an approximation
\begin{equation} \label{fFPGA}
\textbf{F}_{I}^{N} = \textbf{F}_{I} + \mathbf{\Xi }_{I}^{N}, 
\end{equation}
where $\mathbf{\Xi }_{I}^{N}$ is an additive white noise for which
\begin{equation} \label{CrossCorr}
 \left \langle \textbf{F}_{I}\left ( 0 \right ) \mathbf{\Xi } _{I}^{N}\left ( t \right )\right \rangle \cong  0
\end{equation}
holds. Given that $\mathbf{\Xi }_{I}^{N}$ is unbiased, which is in our case is true by its very definition, it is nevertheless possible to accurately sample the Boltzmann distribution by means of a Langevin-type equation \cite{Krajewski,Richters,Karhan}, which in its general form reads as
%Fortunately in our case, it is still possible to accurately obtain Boltzmann sampling by means of a modified Langevin equation 
\begin{equation} \label{LangevinEq}
M_{I}\ddot{\textbf{R}}_{I}=\textbf{F}_{I}+\mathbf{\Xi }_{I}^{N}-\gamma _{N}M_{I}\dot{\textbf{R}}_{I}, 
\end{equation}
where $\dot{\textbf{R}}_{I}$ are the nuclear coordinates (the dot denotes time derivative), $M_I$ are the nuclear masses and $\gamma _{N}$ is a damping coefficient, 
%where \(\textbf{R}_{I}\) are the positions of the atoms, \(M_{I}\) the corresponding atomic nuclear masses and \(\gamma _{N}\) is a friction coefficient 
which is chosen to compensate for \(\mathbf{\Xi }_{I}^{N}\). The latter, in order to guarantee an accurate canonical sampling, has to obey
%\begin{equation}
% \left \langle \textbf{F}_{I}\left ( 0 \right ) \mathbf{\Xi } _{I}^{N}\left ( t \right )\right \rangle \cong  0,
%\end{equation}
the fluctuation-dissipation theorem
\begin{equation}
\left \langle \mathbf{\Xi }_{I}^{N}\left ( 0 \right ) \mathbf{\Xi }_{I}^{N}\left ( t \right ) \right \rangle \cong  2 \gamma_{N} M_I k_{B} T  \delta \left ( t \right ), 
\end{equation}  
where $k_B$ is Boltzmann's constant, and $T$ is the desired temperature. 

Substituting Eq.~\ref{fFPGA} into Eq.~\ref{LangevinEq} results in the desired modified Langevin equation 
\begin{equation} \label{modLangevin}
M_{I}\ddot{\textbf{R}}_{I} = \textbf{F}_{I}^{N}-\gamma _{N}M_{I}\dot{\textbf{R}}_{I}, 
\end{equation}
which will be used throughout the remaining of this paper.

%We further present in this paper the method to effectively compensate for the additive noise introduced on the force using the existing langevin dynamics framework. 
